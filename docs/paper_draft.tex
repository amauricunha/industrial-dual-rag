\documentclass[conference]{IEEEtran}
\usepackage[utf8]{inputenc}
\usepackage[portuguese]{babel}
\usepackage{cite}
\usepackage{amsmath,amssymb,amsfonts}
\usepackage{graphicx}
\usepackage{textcomp}
\usepackage{xcolor}

\title{Diagnóstico Ciber-Físico em Tempo Real: \\ Uma Abordagem de RAG com Contexto Dinâmico}

\author{\IEEEauthorblockN{Seu Nome}
\IEEEauthorblockA{\textit{Departamento de Automação e Sistemas} \\
\textit{Universidade Federal de Santa Catarina (UFSC)}\\
Florianópolis, Brasil \\
seu.email@posgrad.ufsc.br}
}

\begin{document}

\maketitle

\begin{abstract}
Sistemas de diagnóstico industrial baseados em Grandes Modelos de Linguagem (LLMs) frequentemente sofrem de alucinações ou geram recomendações genéricas por falta de acesso ao estado atual do equipamento. Este trabalho propõe uma arquitetura de Geração Aumentada por Recuperação (RAG) de Contexto Dual, aplicada a um Torno Mecânico. O sistema funde documentos estáticos (manuais técnicos e normas ISO) com fluxos de telemetria MQTT em tempo real. Implementou-se um protótipo capaz de alternar entre inferência na nuvem (Groq/Gemini) e local (Llama 3 via Docker). Experimentos simulando falhas críticas, como vibração excessiva e eixos empenados, demonstram que a fusão de contextos permite diagnósticos acionáveis e seguros, superando abordagens de LLM puro.
\end{abstract}

\begin{IEEEkeywords}
RAG, Indústria 4.0, LLM, MQTT, Manutenção Preditiva.
\end{IEEEkeywords}

\section{Introdução}
A operação de máquinas rotativas, como tornos, exige monitoramento constante. Operadores nem sempre conseguem correlacionar rapidamente leituras de sensores com tabelas de normas técnicas. O objetivo deste projeto é criar um assistente cognitivo que receba perguntas em linguagem natural (ex: "A máquina está segura?") e responda cruzando dados ao vivo com limites de engenharia.

\section{Metodologia}
A arquitetura proposta divide-se em três camadas:
\begin{enumerate}
    \item \textbf{Camada Física Simulada:} Um script Python gera dados sintéticos de falhas comuns em tornos (Peça solta, Eixo empenado, RPM excessivo) e transmite via protocolo MQTT.
    \item \textbf{Camada de Contexto:}
    \begin{itemize}
        \item \textit{RAG Estático:} Utiliza ChromaDB para indexar manuais técnicos simulados baseados na norma ISO 10816.
        \item \textit{RAG Dinâmico:} Um ouvinte MQTT mantém o último estado válido da máquina em memória.
    \end{itemize}
    \item \textbf{Camada de Inferência:} Uma API FastAPI orquestra a montagem do prompt e envia para LLMs. O sistema suporta execução local via Ollama (Docker) para privacidade de dados.
\end{enumerate}

\section{Resultados Preliminares}
O sistema foi capaz de identificar corretamente o cenário "PECA\_SOLTA", alertando que a vibração de 8.5 mm/s excedia o limite de 7.1 mm/s da Zona D (Dano), recomendando parada imediata. O modelo sem o contexto dinâmico apenas descreveu genericamente o que é vibração.

\section{Conclusão}
A arquitetura de Contexto Dual prova ser eficaz para trazer confiabilidade a assistentes de IA no chão de fábrica.

\end{document}